\begin{figure}[htbp] 

~\hrule~
 
%\noindent\begin{minipage}{0.5\textwidth} 
{\small 
\begin{tabular}{l@{~~~}l@{~~~}r@{~~~}r@{~~~}c}
\arrayrulecolor{lightgray}
\textbf{Rank} & \textbf{Treatment} & \textbf{50th} & \textbf{25-75th} & \\\hline
  1 &        XTREES &    0.42  &  0.17 & \quart{0}{10}{0}{20} \\
\hline  2 &      CD+FS &    1.03  &  0.69 & \quart{30}{46}{42}{20} \\
  2 &          BIC &    1.04  &  0.01 & \quart{42}{0}{42}{20} \\
  2 &      CD &    1.11  &  1.24 & \quart{18}{80}{46}{20} \\
\hline 
\end{tabular}}

~\\

In this figure, {\bf 50th}
is the median value and  {\bf 25-75th} is the second and third quartile range.
The right-hand-side horizontal lines show the {\bf 50th} median value (as a black dot) in the middle
of the {\bf 25-75th} percentile range (as lines). All values here are expressed as ratios
of the median values seen in the untreated data, so {\em smaller} values are {\em better}).
A median {\bf 50th} value of
1.0 means ``no change from
baseline''; numbers less than 1.0 indicate some reduction to the baseline;
and numbers more than 1.0 indicate ``optimization failure''.
The {\bf Rank} column on the left-hand-side show the results of a statistical
analysis and results ranked ``1'' are best. Results change {\bf Rank} if they are significantly different and pass
 a post-hoc effect size test.
 
 ~\hrule~
\caption{Results: Methods 1,2,3,4 applied to some ground-truth data.
Values collected from   20 repeated runs of each method with different random seeds.
}\label{fig:conf}
\end{figure}